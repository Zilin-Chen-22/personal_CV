%----------------------------------------------------------------------------------------
%	PACKAGES AND OTHER DOCUMENT CONFIGURATIONS 
%----------------------------------------------------------------------------------------
 
\documentclass{resume} % 使用自定义简历样式 
\usepackage[dvipsnames]{xcolor}
\usepackage{ctex}
\usepackage{enumitem}
\usepackage[left=0.4 in,top=0.3 in,right=0.4 in,bottom=0.3in]{geometry} % 页面边距
\newcommand{\tab}[1]{\hspace{.2667\textwidth}\rlap{#1}}
\newcommand{\itab}[1]{\hspace{0em}\rlap{#1}}
\name{陈子林} % 姓名 
\address{清华大学学生公寓28号楼235室,北京,100084} % 地址 
\address{(+86)189-9550-1708 \\ chenzili22@mails.tsinghua.edu.cn} % 联系方式

\definecolor{TsinghuaPurple}{cmyk}{0.58,0.90,0,0}
\renewenvironment{rSection}[1]{
\sectionskip
\textcolor{TsinghuaPurple}{\MakeUppercase{#1}}
\sectionlineskip
\hrule
\begin{list}{}{
\setlength{\leftmargin}{0em}
}
\item[]
}{
\end{list}
}


\begin{document}  

%----------------------------------------------------------------------------------------
%	教育背景
%----------------------------------------------------------------------------------------

\begin{rSection}{教育背景}

{\bf 清华大学机械工程系,北京,中国} \hfill {2022年8月 - 至今}
\\ 
机械工程专业本科在读
\\
当前总GPA(至大三): 3.59/4.0 (前35\%)
\\
核心课程:
\begin{itemize}[itemsep = -0.5em, topsep = -0.5em]
    \item 计算机程序设计基础 (A)
    \item 机械设计基础 (A+)
    \item 大学物理B(2)(A)
    \item 测试与仪器(A-)
\end{itemize}

\vspace{11pt}

{\bf 多伦多大学机械与工业工程系} \hfill {2024年9月 - 2024年12月}
\\ 
机械工程专业交换生,专业课程GPA: 3.9/4.0
\\
主修课程:
\begin{itemize}[itemsep = -0.5em, topsep = -0.5em]
    \item 机械运动学与动力学 (A-)
    \item 流体力学I (A)
    \item 机械系统工程电路基础 (A+)
\end{itemize}

\end{rSection} 

%-------------------------------------------------------------------------------
%	项目经历

\begin{rSection}{项目经历}

\begin{rSubsection}{协作机器人设计 - 研究助理} {2024年2月 - 2024年12月}{导师:汪泽博士,清华大学机械系助理研究员}{}

\item 总结双臂协作机器人运动轨迹规划领域最新研究进展
\item 改进双臂协作机器人末端执行器稳定性
\item \textbf{分析工作空间}及约束条件以防止机械臂碰撞
\item 带领四人团队设计不同构型方案以提高系统刚度

\end{rSubsection}  

%------------------------------------------------

\begin{rSubsection}{机器人轨迹规划 - 研究助理}{2024年5月 - 2024年7月}{}{} 
\item 发表于第九届IEEE高级机器人与机电一体化国际会议,获最佳论文奖
\item 基于Franka机械臂的七自由度机器人轨迹规划
\item 通过图表可视化速度、加速度和加加速度变化
\item 采用多种方案求解通过相同路径点的时间优化问题,并进行多维度对比

\end{rSubsection}

%-------------------------------------------------


\begin{rSubsection}{智能物流车设计 - 项目组长}{2024年7月}{}{}

\item 领导三人团队开发具备自主导航能力的物流运输车
\item 集成循迹导航、避障定位、蓝牙遥控等核心功能
\item 应用PID控制算法实现电机精准控制
\item 采用STM32控制器与Open MV实现计算机视觉与动态路径规划

\end{rSubsection}

%-------------------------------------------------- 
\newpage
\begin{rSubsection}{UTAT无人机竞速队 - 研究助理}{2024年9月 - 2025年1月}{导师:Hugh H.T. Liu(刘泓涛教授),多伦多大学航空航天学院教授,加拿大工程院院士}{}    

\item 参与ESC开关频率与旋转转速速度曲线测试
\item \textbf{构建竞速无人机仿真平台},从开源项目提取控制器模块并移植至机载计算机
\item 使用CasADi求解器进行偏航轴轨迹优化
\item 通过优化轨迹延长闸门识别时间,提升定位精度

\end{rSubsection}

\begin{rSubsection}{人工肌肉驱动无人机系统 - 研究助理}{2025年2月至今}{导师:赵慧婵副教授,清华大学机械工程系党委副书记}{}
    \item 搭建特殊构型无人机仿真系统,并进行测试
    \item 构建人工肌肉驱动\textbf{无人机控制器},并进行前期仿真环境测试
\end{rSubsection}

\end{rSection} 

\begin{rSection}{实习经历}

\begin{rSubsection}{华海清科CMP边缘抛光研究-机械工程师}{2025年6月-2025年7月}{}{}
    \item 对CMP设备抛光过程进行\textbf{运动学仿真建模},搭建全流程可调仿真平台
    \item 优化CMP边缘抛光平台找平流程,极大提高工作效率,简化工作流程
    \item 探究不同工艺参数对于抛光效果的影响,寻找更优抛光参数    
\end{rSubsection}

\begin{rSubsection}{小米科技有限责任公司-机械工程师}{2025年7月至今}{}{}
    \item 协助搭建汽车工厂工业机械臂,进行硬件装配及调试
    \item 进行机械臂\textbf{HIL在环仿真测试},训练工业机械臂进行指定动作,构建工业机械臂控制器,\textbf{开发控制算法}
    \item 开发moveJ、moveL等机械臂所需算法,重构开源项目以方便队友配置环境以及相应硬件使用
\end{rSubsection}

\end{rSection}

%--------------------------------------------------------------------------------------
%  领导力与活动
%--------------------------------------------------------------------------------------

\begin{rSection}{领导力与活动}

\begin{rSubsection}{清华大学学生艺术团交响乐队}{2023年8月 - 2024年7月}{共青团清华大学委员会 - 副队长}{}              
\item 组织参加第七届全国大学生艺术展演获一等奖
\item 负责日常管理与训练安排
\item 策划执行多场音乐会,单场观众超100人,演职人员30人+
\end{rSubsection}  

%------------------------------------------------

% \begin{rSubsection}{“机械力量”品牌支队社会实践支队} {2024年1月 - 2024年3月}{共青团机械系委员会 - 支队长}{} 
% \item 组建16人跨校际团队(含华科、天大各1人)     
% \item 调研多家装备制造企业,与一线工程师深入交流
% \item 获评院系社会实践金奖,个人获评最佳支队长  
% \end{rSubsection}

\end{rSection}
  

%----------------------------------------------------------------------------------------
%	技能与兴趣
%----------------------------------------------------------------------------------------

\begin{rSection}{技能与兴趣}

    \begin{tabular}{ @{} >{\bfseries}l @{\hspace{6ex}} l }  
    语言成绩 & 雅思7.5(阅读8.5 听力8.5 口语6.0 写作6.5)\\
    研究兴趣 & 自动控制与轨迹规划、动力学控制\\    
    实践技能 & 金工实习、工业机器人实验操作\\
    编程语言 & 精通C、Python、MATLAB,熟悉Java、C++ \\
    办公软件 & Microsoft Office、Photoshop、Auto CAD、SolidWorks、Multisim\\
     
    \end{tabular}   
    
    \end{rSection}

%---------------------------------------------------------------------------------
% 获奖情况
%--------------------------------------------------------------------------------


\begin{rSection}{荣誉奖项}
{文艺优秀奖学金}\hfill {\em 2023} \\
{文艺优秀奖学金}\hfill {\em 2024} \\
{社会实践优秀支队长} \hfill {\em 2024春}
\end{rSection}

\end{document}