%----------------------------------------------------------------------------------------
%	宏包和其他文档配置
%----------------------------------------------------------------------------------------

\documentclass{resume} % 使用自定义的 resume.cls 样式
\usepackage{ctex}
\usepackage[dvipsnames]{xcolor}
\usepackage{enumitem}
\usepackage[left=0.4 in,top=0.3 in,right=0.4 in,bottom=0.3in]{geometry} % 文档页边距
\newcommand{\tab}[1]{\hspace{.2667\textwidth}\rlap{#1}}
\newcommand{\itab}[1]{\hspace{0em}\rlap{#1}}
\name{陈子林} % 你的姓名
%\address{北京市清华大学机械工程系28号学生公寓235房间,邮编100084} % 地址
%\address{123 某某街道 \\ 城市,省份 邮编} % 第二地址(可选)
\address{(+86)189--9550--1708 \\ chenzili22@mails.tsinghua.edu.cn} % 电话号码和邮箱

\definecolor{TsinghuaPurple}{cmyk}{0.58,0.90,0,0}
\renewenvironment{rSection}[1]{
\sectionskip
\textcolor{TsinghuaPurple}{\MakeUppercase{#1}}
\sectionlineskip
\hrule
\begin{list}{}{
\setlength{\leftmargin}{0em}
}
\item[]
}{
\end{list}
}

\begin{document}

%----------------------------------------------------------------------------------------
%	教育经历
%----------------------------------------------------------------------------------------

\begin{rSection}{教育背景}

{\bf 清华大学,北京,中国,机械工程学院 } \hfill {2022年8月 -- 至今}
\\ 
机械工程学士
\\
当前平均绩点\@: 3.59/4.0(专业前35\%)
\\
英语水平:雅思总分7.5(阅读8.5,听力8.5,口语6.0,写作6.5)
\\
核心课程:
\begin{itemize}[itemsep = -0.5em, topsep = -0.5em]
    \item 机电系统设计实践(A)
    \item 机械工程力学(1)(A+)
    \item 机器人认知与实践(A-)
\end{itemize}

\vspace{11pt}

{\bf 多伦多大学,机械与工业工程系 } \hfill {2024年9月 -- 2024年12月}
\\ 
机械工程交换生
\\
学期专业课程绩点:3.9/4.0
\\
课程:
\begin{itemize}[itemsep = -0.5em, topsep = -0.5em]
    \item 机构运动学与动力学(A-)
    \item 流体力学I(A)
    \item 面向机械系统的电路应用(A+)
\end{itemize}

\end{rSection}

%-------------------------------------------------------------------------------
%	项目经历
%-------------------------------------------------------------------------------

\begin{rSection}{项目经历}

\begin{rSubsection}{人工肌肉驱动无人机——科研助理}{2025年2月 -- 至今}{导师:清华大学机械工程学院副教授赵会湛}{}
    \item 设计并搭建了一套跨平台的人工肌肉驱动无人机仿真框架,C++开发,Python可视化
    \item 独立开发一款重23克的人工肌肉驱动无人机\textbf{飞控系统},实现稳定飞行和快速控制响应
    \item 为实验室保留了完整的\textbf{仿真框架代码},以支持后续研究
\end{rSubsection}

\begin{rSubsection}{UTAT 无人机竞速队——科研助理}{2024年9月 -- 2025年1月}{导师:多伦多大学空中机器人研究中心主任刘洪涛教授}{}    
    \item 协助测试ESC开关频率与电机响应曲线,提高仿真精度,误差插值后\textbf{控制在5\%以内}
    \item 基于开源代码(阅读betaflight全代码)开发\textbf{无人机竞速仿真器},聚焦飞控逻辑,适配机载计算
    \item 使用CasADi优化偏航轴轨迹,解决\textbf{时间最优路径}问题,提高过门精度和视野,使无人机速度提升\textbf{50\%以上}
\end{rSubsection}

\begin{rSubsection}{智能车设计——团队负责人}{2024年6月 -- 2024年8月}{}{}
\item 带领三人团队设计并制作一款能够沿预定路线搬运物体的自主小车
\item 集成\textbf{循迹、实时避障、目标识别}和蓝牙遥控等功能
\item 实现PID控制精调电机参数,保证速度与转向角度的精确控制
\item 使用STM32微控制器实现嵌入式控制,并通过OpenMV完成\textbf{实时视觉处理、图像识别}及动态路径规划
\end{rSubsection}

\newpage

\begin{rSubsection}{协作机器人设计——科研助理}{2024年2月 -- 2024年12月}{导师:清华大学机械工程学院助理院长王泽}{}
    \item 撰写双臂协作机器人运动轨迹规划方法的调研报告
    \item 提出并实现提高末端执行器动态协作运动稳定性的策略
    \item \textbf{分析并仿真工作空间与关节约束},避免双臂内部碰撞
    \item 参与多种机器人构型设计与评估,提高系统刚度和可靠性
\end{rSubsection}

\end{rSection}

%----------------------------------------------------------------------------------------
%	实习经历
%----------------------------------------------------------------------------------------

\begin{rSection}{实习经历}

\begin{rSubsection}{仿人机器人手臂路径规划与控制HIL测试}{2025年7月 -- 至今}{小米公司,机械工程师}{}
    \item 参与汽车产线中部署的仿人机器人手臂的装配和调试
    \item 进行\textbf{硬件在环(HIL)}仿真,验证机器人控制系统在实时环境下的性能
    \item 开发并测试7自由度机械臂的运动规划控制算法,协助完成\textbf{\textit{ISO 9283:1998}标准测试}
\end{rSubsection}

\begin{rSubsection}{化学机械抛光边缘修整仿真}{2025年6月 -- 2025年7月}{华海清科公司,机械工程师}{}
\item 搭建\textbf{运动学仿真模型},研究CMP边缘修整工艺
\item 通过机械调整优化找平过程,大幅提高速度和稳定性(优化前约\textbf{60分钟},优化后约\textbf{10分钟})
\item 分析关键工艺参数对表面质量和边缘完整性的影响,优化操作规范
\end{rSubsection}

\end{rSection}

%--------------------------------------------------------------------------------------
%   校园活动和领导力
%--------------------------------------------------------------------------------------

\begin{rSection}{领导力与活动}

\begin{rSubsection}{清华大学交响乐团}{2023年8月 --- 2024年7月}{副团长,共青团清华大学委员会}{}              
\item 统筹安排乐团参加第七届全国大学生艺术展演并获得\textbf{国家一等奖}
\item 组织排练计划及内部事务,筹办多场高规格音乐会,参与人员30余人
\item 担任\textbf{低音提琴首席及钢琴手},兼顾演奏与管理工作
\end{rSubsection}  

\end{rSection}

%----------------------------------------------------------------------------------------
%	技能与兴趣
%----------------------------------------------------------------------------------------

\begin{rSection}{技能与兴趣}

\begin{tabular}{ @{} >{\bfseries}l @{\hspace{6ex}} l }  
研究兴趣 & 自主控制系统、最优轨迹规划、机器人动力学 \\    
实践技能 & 金属加工、机加操作、工业机器人实验 \\    
编程语言 & 精通C、C++、Python、MATLAB,熟悉Java \\    
软件能力 & AutoCAD、SolidWorks、Multisim、Adobe Photoshop \\
\end{tabular}   

\end{rSection}

%---------------------------------------------------------------------------------
%  奖项
%--------------------------------------------------------------------------------

\begin{rSection}{获奖情况} \itemsep -2pt
{文艺奖学金}\hfill {\em 2023、2024} \\
{最佳社会实践团队负责人}\hfill {\em 2024年春季}
\end{rSection}

\end{document}
